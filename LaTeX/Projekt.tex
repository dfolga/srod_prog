\documentclass[12pt,a4paper,titlepage]{report}

\usepackage{polski}
\usepackage[utf8]{inputenc} 

\author{Damian Folga}

\title{Projekt środowisko programisty.}

\begin{document}

\maketitle
\tableofcontents
\newpage
\chapter{Pierwszy rozdział}
\section{Pierwszy podrozdział}
Treść pierwszego rozdziału.
Creating a footnote is easy.\footnote{An example footnote.}

\begin{tabular}{|r|l|} \hline
7C0 & heksadecymalnie \\
\hline \hline
3700 & oktalnie \\
\hline \hline
11111000000 & binarnie \\
\hline \hline
1984 & dziesietnie \\ \hline

\end{tabular}

\section{Drugi podrozdział}
\section{Trzeci podrozdział}
\newpage
\chapter{Drugi rozdział}
\section{Pierwszy podrozdział}
\begin{math}
 $$
\lim_{x \rightarrow 0} \frac{\sin x}{x}=1
$$
\end{math}

\section{Drugi podrozdział}
\section{Trzeci podrozdział}
\cite{pa}Treść drugie rozdziału.

\newpage
\chapter{Trzeci rozdział}
\section{Pierwszy podrozdział}
\section{Drugi podrozdział}
\section{Trzeci podrozdział}
Treść trzeciego rozdziału.
\newpage
\chapter{Czwarty rozdział}
\section{Pierwszy podrozdział}
\section{Drugi podrozdział}
\section{Trzeci podrozdział}
Treść czwartego rozdziału.
\newpage
\chapter{Piąty rozdział}
\section{Pierwszy podrozdział}
\section{Drugi podrozdział}
\section{Trzeci podrozdział}
Treść piątego rozdziału.
\newpage
\chapter{Szósty rozdział}
\section{Pierwszy podrozdział}
\section{Drugi podrozdział}
\section{Trzeci podrozdział}
Treść szóstego rozdziału.
\newpage
\begin{thebibliography}{99}
\addcontentsline{toc}{section}{Bibliografia}
\bibitem{pa} H.~Partl:
\emph{German \TeX},
TUGboat Vol.~9,, No.~1 ('88)
\bibitem{pa} H.~Partl:
\emph{German \TeX},
TUGboat Vol.~9,, No.~1 ('88)
\bibitem{pa} H.~Partl:
\emph{German \TeX},
TUGboat Vol.~9,, No.~1 ('88)
\bibitem{pa} H.~Partl:
\emph{German \TeX},
TUGboat Vol.~9,, No.~1 ('88)
\end{thebibliography}
\end{document}
