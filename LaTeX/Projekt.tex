\documentclass[12pt,a4paper,titlepage]{report}

\usepackage{polski}
\usepackage[utf8]{inputenc} 

\author{Damian Folga}

\title{Metody przydzielania mandatów wyborczych.}

\begin{document}

\maketitle
\tableofcontents
\newpage
\chapter{Metoda D'Hondta}
\textbf{Metoda D’Hondta} (również: \textit{Jefferson’s method, Bader-Ofer method}) – metoda stosowana do podziału mandatów w systemach wyborczych opartych na proporcjonalnej reprezentacji z listami partyjnymi. Jej nazwa pochodzi od nazwiska belgijskiego matematyka Victora D’Hondta.
\section{Podział mandatów}
W metodzie tej dla każdego komitetu wyborczego, który przekroczył próg wyborczy, obliczane są kolejne ilorazy całkowitej liczby głosów uzyskanych przez dany komitet i kolejnych liczb naturalnych, czyli ilorazy wyborcze. O podziale miejsc pomiędzy komitetami decyduje wielkość obliczonych w ten sposób ilorazów. Można to przedstawić wzorem:

\begin{math}
I_i=\frac{G}{i}
\end{math} \\
gdzie:

\begin{math}
I_i-i
\end{math}
-ty iloraz wyborczy,

\begin{math}
G-
\end{math}
-całkowita liczba głosów oddana na dany komitet w wyborach,

\begin{math}
i-
\end{math}
liczba naturalna,
\begin{math}
i\geq{1}
\end{math}

Tak więc dla każdego komitetu liczba uzyskanych głosów jest dzielona kolejno przez 
\begin{math} 1,2,3,\dots ,n\end{math}. W ten sposób uzyskuje się malejące wielkości 
\begin{math}
I
\end{math}
, które porównywane są następnie z wynikami wszystkich komitetów biorących udział w wyborach i szeregowane w kolejności od największej do najmniejszej. Mandaty przydziela się zgodnie z określoną w ten sposób kolejnością, poczynając od najwyższego wyniku do najniższego, aż do momentu, gdy liczba dostępnych miejsc zostanie wyczerpana.
\newpage
\section{Przykład}
Mamy komitety A, B i C, które otrzymały kolejno 720, 300 i 480 głosów. Do obsadzenia jest 8 mandatów. \\ \\
\textbf{1 krok}: obliczenie ilorazów \\ \\
\begin{tabular}{|l|l|l|l|} \hline
Dzielnik & Komitet A & Komitet B & Komitet C \\
\hline
1 & 720 (pierwszy mandat) & 300 (czwarty) & 480 (drugi) \\
\hline
2 & 360 (trzeci) & 150 & 240 (szósty) \\
\hline
3 & 240 (piąty) & 100 & 160 (ósmy) \\ 
\hline
4 & 180 (siódmy) & 75 & 120 \\ \hline
5 & 144 & 60 & 96 \\
\hline

\end{tabular}\\ \\
\textbf{2 krok}: ułożenie ilorazów w kolejności malejącej (w nawiasach komitet):
\begin{enumerate}
\item (A) – 720
\item (C) – 480
\item (A) – 360
\item (B) – 300
\item (A) – 240
\item (C) – 240
\item (A) – 180
\item (C) – 160
\end{enumerate} 
itd. \\ \\
W związku z tym, że do rozdzielenia jest 8 mandatów, 4 mandaty otrzymuje komitet A (ilorazy 720, 360, 240 i 180), 1 mandat – komitet B (iloraz 300) oraz 3 mandaty – komitet C (ilorazy 480, 240 i 160). \\ \\
W przypadku gdyby kilka komitetów uzyskało takie same ilorazy stosuje się różne metody dodatkowego szeregowania. W Polsce wybrano następujący sposób – jeżeli kilka list uzyskało ilorazy równe ostatniej liczbie z liczb uszeregowanych w podany sposób, a list tych jest więcej niż mandatów do rozdzielenia, pierwszeństwo mają listy w kolejności ogólnej liczby oddanych na nie głosów. Gdyby na dwie lub więcej list oddano równą liczbę głosów, o pierwszeństwie rozstrzyga liczba obwodów głosowania, w których na daną listę oddano większą liczbę głosów.
\section{Zbliżenie idealnej proporcjonalności}
Doskonała proporcjonalność nie zawsze jest możliwa. Metody reprezentacji proporcjonalnej podchodzą do jej przybliżenia na różne sposoby, które implikują różne koncepcje nieproporcjonalności. Metoda D’Hondta minimalizuje największy współczynnik korzyści, 

\begin{math}max_{k}w_{k}={\frac {m_{k}}{g_{k}}}\end{math}, \\ \
gdzie: \\
\begin{math}w_{k}-\end{math}współczynnik korzyści komitetu
\begin{math}k,\end{math} \\
\begin{math}m_{k}-\end{math}udział mandatów udzielonych do komitetu
\begin{math}k,\end{math} \\
\begin{math} m_{k}\in [0,1],\;\sum\limits _{k}m_{k}=1,\end{math} \\
\begin{math}g_{k}-\end{math}udział głosów oddanych na komitet
\begin{math}k,\end{math} w wyborach,\\
\begin{math} g_{k}\in [0,1],\;\sum\limits_{k} g_{k}=1.\end{math}[1] \\
Metoda D’Hondta dzieli głosy na dokładnie proporcjonalnie reprezentowane i niereprezentowane, minimalizując udział niereprezentowanych głosów 

\begin{math}\pi ^{*}=1-{\frac {1}{\max _{k}w_{k}}}\end{math}[2] \\
Niereprezentowany udział głosów komitetu jest 

\begin{math} n_{k}=g_{k}-(1-\pi ^{*})m_{k},\;n_{k}\in [0,g_{k}],\sum _{k}\,n_{k}=\pi ^{*}\end{math}[2] \\ \\
Przy minimalizacji ogólnej liczby niereprezentowanych głosów metoda D’Hondta bierze pod uwagę tylko największy współczynnik korzyści. Jeśli do oceny proporcjonalności stosuje się współczynnik korzyści, wynika to z tego że metoda D’Hondta faworyzuje duże ugrupowania w większym stopniu niż druga spośród najpopularniejszych metod przeliczania głosów – metoda Sainte-Laguë.
\section{Stosowanie}
Metoda D’Hondta jest najczęściej stosowaną metodą reprezentacji proporcjonalnej w wyborach do parlamentów narodowych[3]. Stosuje się ją przy podziale mandatów w wyborach parlamentarnych m.in. w Austrii, Finlandii, Izraelu, Holandii i Hiszpanii. W Polsce stosowano ją m.in. w parlamentarnych ordynacjach wyborczych II Rzeczypospolitej (do 1935 r.), a także w III Rzeczypospolitej (z wyłączeniem wyborów w 1991 r. oraz wyborów w 2001 r.) przy podziale mandatów do Sejmu oraz w wyborach samorządowych (do rad gmin powyżej 20 000 mieszkańców[4], rad powiatów oraz sejmików województw).

W Izraelu metoda ta jest w użyciu od 1973, przy czym znana jest pod nazwą Bader-Ofer method od nazwisk parlamentarzystów, którzy zaproponowali jej wprowadzenie (Jochanan Bader i Awraham Ofer)[5].
\newpage
\chapter{Metoda}
\section{Pierwszy podrozdział}


\section{Drugi podrozdział}
\section{Trzeci podrozdział}
\cite{pa}Treść drugie rozdziału.

\newpage
\chapter{Trzeci rozdział}
\section{Pierwszy podrozdział}
\section{Drugi podrozdział}
\section{Trzeci podrozdział}
Treść trzeciego rozdziału.
\newpage

\begin{thebibliography}{99}
\addcontentsline{toc}{section}{Bibliografia}
\bibitem{pa} H.~Partl:
\emph{German \TeX},
TUGboat Vol.~9,, No.~1 ('88)
\bibitem{pa} H.~Partl:
\emph{German \TeX},
TUGboat Vol.~9,, No.~1 ('88)
\bibitem{pa} H.~Partl:
\emph{German \TeX},
TUGboat Vol.~9,, No.~1 ('88)
\bibitem{pa} H.~Partl:
\emph{German \TeX},
TUGboat Vol.~9,, No.~1 ('88)
\end{thebibliography}
\end{document}
